% +--------------------------------------------------------------------+
% | Sample Chapter
% +--------------------------------------------------------------------+

\cleardoublepage



\chapter{Introducción y objetivos.}
\label{introduccion}

%explicar aqui el problema del doble gasto
En 1969 el Departamento de Defensa de los Estados Unidos instaló los primeros nodos de la \textit{Advanced Research Projects Agency Network} (ARPANET), precursora de la actual Internet. Desde este momento surgió la necesidad de garantizar las comunicaciones entre varias computadoras incluso en el supuesto en que algunas de ellas dejaran de funcionar, o se interrumpieran parte de las conexiones que hacían posible la transmisión de información\citep{arpanet}. Nacen así las redes de ordenadores y con ellas el concepto de sistemas distribuidos.

Desde este momento surgieron diversas aplicaciones y protocolos, tales como Napster, Gnutella o Bittorrent \citep{redes_p2p}, que buscaban resolver diferentes problemas de forma cooperativa y distribuida. El sector financiero no estuvo ajeno a estas transformaciones, y ya desde 1983 se teorizó sobre la posibilidad de de utilizar monedas digitales, aunque requiriendo de cierta entidad central que validara las operaciones \citep{chaum1983blind}. También surgió la idea con B-Money \citep{b_money} de utilizar una moneda digital completamente descentralizada, pero esta propuesta no llegó a ser implementada.


 %A este registro lo llamaremos cadena de bloques para distinguirlo del nombre del protocolo.
%aquí poner algo de la historia del protocolo, hablar del blockchain y de las distintas versiones (fundamentalmente criptomonedas) así enlazo con lo ya escrito.

En el año 2008, con la publicación del artículo \textit{Bitcoin: A Peer-to-Peer Electronic Cash System} \citep{bitcoin}, nace el protocolo Blockchain. Se desconoce la verdadera identidad de la persona u organización oculta bajo el seudónimo de Satoshi Nakamoto, autor de este documento. %\footnote{https://bitcoin.org/bitcoin.pdf}
En el año 2009, el propio Nakamoto publicó el código, escrito en C++, de la criptomoneda Bitcoin. Posteriormente fueron creadas nuevas monedas digitales basadas en la idea original del Bitcoin y a partir del año 2014 se comenzaron a estudiar aplicaciones del Blockchain distintas de las criptomonedas.

Se puede definir entonces al Blockchain como un protocolo que permite mantener un registro distribuido e inmutable de las transacciones o comunicaciones entre los participantes de una determinada red. Y que para garantizar en todo momento que la información almacenada es veraz, se tiene también un mecanismo de consenso que no necesita de ningún tipo de centralización para funcionar. En este punto conviene hacer un apunte sobre la terminología usada, aunque en la literatura se suele utilizar \textit{blockchain} en minúsculas para hacer referencia al registro donde se almacena la información y reservar \textit{Blockchain} para hacer referencia a la tecnología o al protocolo, en este trabajo utilizaremos la traducción al español: \textit{cadena de bloques} cuando se haga referencia a la estructura que conforma la base de datos.

Una aspecto importante de las primeras criptomonedas es su carácter completamente público: cualquier usuario de internet podía unirse a la red y participar activamente sin necesidad de ninguna autorización o verificación. 

Todas estas implementaciones, independientemente de las variaciones en el protocolo, tienen en común el carácter distribuido del algoritmo y la inmutabilidad del registro (cadena de bloques). Sin embargo, se pueden establecer 3 distinciones generales en base a como se trate el aspecto público y el nivel de  descentralización \citep{type_blockchain}:
\begin{itemize}
\item \textit{Blockchain público:} Cualquiera puede unirse a la red, todos los nodos pueden leer el registro, realizar transacciones y participar en el consenso.
\item \textit{Blockchain de consorcio:} En el algoritmo de consenso solo intervienen una parte de los nodos. El acceso a la red y las consultas al registro en principio están abiertas a todos. Este sería el caso de una red pública y centralizada.
\item \textit{Blockchain privado:} El acceso a la red no es público y además tanto la lectura del registro como la participión en el algoritmo de consenso pueden encontrarse limitados. Estamos ante una red privada que puede estar en mayor o menor medida centralizada.
\end{itemize}

%Debe notarse que la frontera entre un protocolo blockchain de consorcio y uno privado son más bien difusas
Dado que es la variante más conocida y utilizada, y es a la que pertenece la primera criptomoneda, el Bitcoin, en este trabajo nos centraremos fundamentalmente en el Blockchain de tipo público. 


Al igual que no existe una única alternativa en la elección de los niveles de centralización y privacidad, el mecanismo utilizado para alcanzar un acuerdo entre los participantes en la red (consenso) difiere entre las diferentes implementaciones del Blockchain. Sin embargo, para explicar las diferencias entre cada uno de estos algoritmos se deben definir y estudiar ciertas cuestiones referentes a la criptografía.% y curvas algebraicas.

Será por tanto el primer objetivo de este trabajo dar las nociones matemáticas básicas que fundamentan el Blockchain. De eso tratará el capítulo \ref{cap1}. Una vez sea posible justificar parte del marco teórico de esta tecnología el siguiente paso será explicar su funcionamiento, tarea que será realizada en el tema \ref{cap3}. En este punto se explicarán de forma más extensa las diferencias entre algunas de las diferentes implementaciones de este protocolo existentes que han sido mencionadas de forma somera en esta introducción. Hay que tener en cuenta que el desarrollo de este protocolo ha estado estrechamente ligado a las distintas implementaciones del mismo (o más bien al éxito de estas) y no tanto a un trabajo teórico sistemático. 

Un tercer objetivo consistirá en estudiar el Blockchain dentro del contexto específico de la computación distribuida y de la resolución de uno de los problemas fundamentales de este campo. Sobre este asunto versará el tema \ref{cap4}. Para finalizar, se realizará en el capítulo \ref{implementacion} una implementación del Blockchain en la que se podrá comprobar que es posible llevar a la práctica, de forma relativamente sencilla, las ideas desarrolladas en este trabajo.

%Más allá del interés socioeconómico que puedan despertar las criptomonedas el blockchain constituye una potente herramienta dentro de la programación distribuida

%siempre que garanticen que la información que cada nodo pretende escribir en el registro es considerada cierta por una parte significativa (quorum) de los otros nodos. Una característica de las primeras implementaciones del protocolo blockchain es que estaban pensadas para trabajar con grandes cantidades de participantes, por tanto es infactible que el quorum sea toda la red.

%hablar de proof of work, algoritmos de consenso, y porque para las criptos se usan estos y no por ejemplo raft y demás. Ver por que no se pueden usar raft y paxos para blockchain (esto quizas iria mejor en otro tema)
%revisar en el primer parrafo del capitulo lo que se dice de este algoritmo
%meter aqui algo del problema de los generales bizantinos, explicar el problema y demás, y decir que sobre esto se volverá al final del trabajo.
%no entrar en mucho detalle, en siguientes capítulos esto se tocará más.






