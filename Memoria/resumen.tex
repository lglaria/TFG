% +--------------------------------------------------------------------+
% | Copyright Page
% +--------------------------------------------------------------------+

\newpage

\thispagestyle{empty}

\begin{center}

{\bf \Huge Resumen}

  \end{center}
\vspace{1cm}

Desde la creación del Bitcoin en 2008 y del crecimiento de las inversiones en criptomonedas a partir de 2012, el protocolo Blockchain ha ido despertando cada vez más interés por sus posibles aplicaciones en diversos campos.

En este trabajo nos alejaremos de las cuestiones económicas relacionadas con las criptomonedas para por un lado analizar los fundamentos matemáticos de esta tecnología y por otro lado ver la relación de este protocolo con ciertos problemas de la programación distribuida. Para ello se estudia el funcionamiento y corrección de una posible especificación del Blockchain. También se ha hecho una implementación de este protocolo donde se han podido llevar a la práctica algunas ideas desarrolladas en el trabajo.
%Una de las tecnologías que han despertado un mayor interés recientemente es el blockchain.

%En cada uno de los capítulos de este trabajo se tocará 
\vspace{1cm}

% +--------------------------------------------------------------------+
% | On the line below, repla	ce Fecha
% |
% +--------------------------------------------------------------------+

\begin{center}

{\bf \Large Palabras clave}

   \end{center}

   \vspace{0.5cm}
   
   Blockchain, Criptografía de clave pública, Bitcoin, Funciones Hash, Consenso, Prueba de trabajo, Sistemas distribuidos.
   




