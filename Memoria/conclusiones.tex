\cleardoublepage

\chapter{Conclusiones}\label{conclusiones}
\section{Discusión plan de estudios}
En la realización de este trabajo han sido importantes los conocimientos adquiridos en las asignaturas de Álgebra Básica y Ecuaciones Algebraicas del plan de estudios de Matemáticas. El contenido de estas materias resultó esencial para entender las diversas cuestiones referentes a la criptografía tratadas en el capítulo \ref{cap1}. Sin embargo, todo lo relativo a curvas elípticas se estudia en la asignatura de Curvas Algebraicas, optativa del último curso del itinerario de Matemática Pura, que por tanto no es común a todos los estudiantes. Igualmente, no existe en la carrera ninguna asignatura donde se traten en profundidad la criptografía, que tal y como se ha visto no solo es un área de conocimiento fundamental en el desarrollo del Blockchain, sino que resulta fundamental en multitud de áreas de la economía, la ciencia o la industria y a su vez está conectada directamente con el álgebra.

Al problema del consenso tratado en \ref{cap4} se hace mención en la asignatura de Programación Paralela, optativa del itinerario de Ciencias de la Computación, pero no es este tema uno de los contenidos centrales de esta materia.

En lo referente a la implementación del protocolo Blockchain han sido de gran importancia los conocimientos adquiridos en la asignatura de Informática de primer curso, sobre el lenguaje de programación Python. Han sido también de gran importancia la optativa de cuarto curso Programación Paralela, mencionada anteriormente, donde se estudian casos prácticos de sistemas distribuidos. Esto ha hecho posible construir la red \textit{P2P} sobre la que se ha establecido la  implementación del protocolo Blockchain realizada en este trabajo. Al igual que no existe una asignatura donde se traten cuestiones teóricas referentes a la criptografía, tampoco existe una donde estas cuestiones se analicen desde un punto de vista más práctico, algo que también se ha echado en falta en el proceso de implementación, por lo que se ha tenido que recurrir a diferentes manuales y documentos mencionados en la bibliografía.  
\section{Ideas finales y posible trabajo futuro}
Este trabajo ha permitido desarrollar los conceptos fundamentales en los que se sustenta el protocolo Blockchain. Ha sido posible justificar su funcionamiento y corrección a partir de la teoría de curvas algebraicas, en particular del algoritmo ECDSA para la firma y verificación de mensajes, así como de las funciones hash. El interés despertado por esta tecnología en el último lustro se ha debido fundamentalmente a las oportunidades de negocio que han generado las criptomonedas. Por tanto una parte importante de los trabajos y documentos sobre el blockchain se han centrado en como generar valor a partir de este protocolo pasando por alto los aspectos teóricos que sustentan su funcionamiento. 
También suele ser común tratar los términos Blockchain y Bitcoin como sinónimos, esto además de no ser exacto, pues como hemos visto el Blockchain trasciende a esta criptomoneda, enturbia a este protocolo al vincularlo con ciertas actividades ilegales (o al menos alegales) que se han aprovechado del anonimato y seguridad que ofrece el Bitcoin.


Por otra parte se han conseguido explicar cuestiones generales sobre la especificación del Blockchain que resultan de utilidad en el estudio de aplicaciones de esta tecnología. Entender este protocolo como un mecanismo de consenso permite utilizarlo en la resolución de diversos problemas de la programación distribuida. Por último, la implementación del protocolo además de mostrar uno de los posibles usos que se le puede dar a esta tecnología prueba que llevar a la práctica las ideas básicas del Blockchainn es una tarea directa y sencilla.


Las reservas o temores de diversos organismos entre los que se encuentran los estados respecto a las criptomonedas están bien fundadas en la medida en que la estructura descentralizada de cadena de bloques anula el control de las entidades centrales sobre el sistema monetario. Pero más allá de las criptomonedas el Blockchain tiene el potencial para introducir transformaciones en diversos aspectos de la sociedad. Esta tecnología cambiará la forma de entender los contratos, las transacciones y en general todas las relaciones que requieren de cierta confianza entre las partes y que tradicionalmente han necesitado de la intervención de intermediarios que actúen como garantes del buen funcionamiento de tales actividades. En cualquier caso, hablar del Blockchain como una (o la) tecnología del futuro tampoco es exacto, porque ya es una tecnología del presente.


%Esta tecnología ha despertado un enorme interés en el último lustro y poder entender y justificar su funcionamiento resulta de gran importancia. La importancia de las funciones hash, y el 



%Las oportunidades de negocio que han generado las criptomonedas ha despertado el interés de las escuelas de negocio y facultades de economía, sin embargo el acercamiento de matemáticos y científicos de la computación al blockchain ha sido mucho más discreto. Por tanto, aún existen pocos artículos y trabajos teóricos sobre el tema.
%La teoría de curvas elípticas es un campo muy desarrollado y de gran importancia tanto en el álgebra como en la criptografía. 

%Más allá de las criptomonedas el blockchain tiene el potencial de introducir profundas transformaciones tanto económicas como sociales. Estas transformaciones

%Hablar del potencial del blockchain resulta contraproducente pues

%La trascendencia de esta tecnología va más allá de una aplicación ec