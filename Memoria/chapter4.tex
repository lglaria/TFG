\cleardoublepage

% +--------------------------------------------------------------------+
% | Replace "This is Chapter 3" below with the title of your chapter.
% | LaTeX will automatically number the chapters.
% +--------------------------------------------------------------------+
\chapter{El problema del consenso}
\label{cap4}
%Explicar el problema de los generales bizantinos...algoritmos paxos y raft, algo de historia

Uno de los problemas fundamentales dentro de la computación distribuida es el de mantener o alcanzar alguna clase de acuerdo entre los miembros de determinada red. Una de las definiciones posibles de un sistema distribuido nos dice que son colecciones de ordenadores que ante sus usuarios funcionan como una única máquina \citep{distributed_system}, así, el hecho de disponer de algún mecanismo de consenso es fundamental para que el sistema se comporte de forma adecuada. Algo más cerca de la idea de consenso que nos interesa de cara al blockchain están los acuerdos que se alcanzan en una base de datos entre todos los procesos antes de hacer un \textit{commit}, donde deben decidir si abortar o no la transacción con la certeza de que la acción tomada será la misma para todos.

En un sistema distribuido genérico suponiendo que todos los miembros actúan siempre de acuerdo a ciertas reglas comunes (no hay comportamientos maliciosos ni fallos), las máquinas de los participantes no se desconectan de la red de forma imprevista y que los canales de comunicación son estables, es decir, todo mensaje llega a su destinatario de forma íntegra y a lo sumo en tiempo $t$, el consenso se puede alcanzar siempre de forma sencilla con algún protocolo como el commit de 2 fases \citep{2commit}. En la práctica no se suelen cumplir alguna o varias de estas condiciones, así que el interés está en crear algoritmos robustos que puedan funcionar bajo múltiples condiciones adversas.

%Aunque a lo largo del texto se haya hablado de algoritmo de consenso para referirse al algoritmo de prueba de trabajo (o a sus alternativas)
El protocolo blockchain descrito en \ref{cap3} es de hecho un algoritmo de consenso. Queda por ver las limitaciones de esta clase de consenso y como se podría aplicar a un contexto más general.
\section{El problema de los generales bizantinos}
El problema de los generales bizantinos \citep{byzantine generals} plantea un experimento mental en el que un grupo de divisiones de un ejército (del Imperio de Bizancio) se encuentran asediando una ciudad. Los generales al frente de cada división deben acordar, en una versión simplificada del problema, si atacan la ciudad o se retiran. Las líneas de comunicación no son seguras, así que algunos mensajes pueden no llegar a su destino. Igualmente, entre los generales existen traidores que pueden no tomar la mejor decisión o comunicar una acción y realizar otra diferente.
El problema es por tanto alcanzar un acuerdo entre los generales leales sobre que decisión tomar. Para garantizar el éxito la decisión tomada debe ser la misma para todos ellos.

Este es claramente un problema de consenso como los definidos al inicio de este capítulo. En general, para alcanzar un acuerdo se necesitan al menos $3m+1$ generales en total si tenemos $m$ generales traidores \citep{byzantine generals}, así por ejemplo con 3 generales si uno de ellos es traidor no existiría solución.

El algoritmo \textit{Paxos}, creado por  Leslie Lamport resuelve el problema anterior \citep{paxos}. El algoritmo \textit{Raft} se considera una simplificación de \textit{Paxos} usando métodos e ideas similares \citep{raft}. Tanto \textit{Paxos} como \textit{Raft} para alcanzar el consenso requieren de la elección de un líder.

%\textit{Plantear el problema, explicar las soluciones existentes: algoritmos paxos, raft... }
\section{Solución desde el Blockchain}
El problema anterior puede plantearse desde la perspectiva del protocolo blockchain. Los nodos o participantes de la red serán los generales y el acuerdo que buscan estará reflejado en la cadena de bloques. Algunos nodos toman una decisión (atacar o retirarse) y se la comunican a los otros participantes. Cada nodo \textit{trabaja} en el primer mensaje que reciba y con el que esté de acuerdo: si por ejemplo considera que lo mejor es retirarse \textit{trabajará} en el primer \textit{retirarse} que reciba. Aquí trabajar es buscar un nonce del bloque que contiene las palabras \textit{retirarse} o \textit{atacar}, junto con información adicional como un índice y un timestamp por ejemplo, de longitud fijada por el protocolo. Cuando un nodo resuelve este problema comunica a los demás su solución y todos (los que estén de acuerdo con esta decisión) trabajan en un nuevo bloque que tenga como predecesor esta respuesta. Pasado cierto tiempo una de las dos cadenas (la de \textit{atacar} o la de \textit{retirarse}) habrá alcanzado cierta longitud fijada en el protocolo y por tanto se puede considerar que son mayoría quienes apoyan realizar la acción definida en esa cadena.

Puede darse el caso que la decisión a tomar no esté clara y por tanto los generales leales se encontrarán divididos en dos grupos de tamaño similar. En ese caso la decisión dependerá de los mensajes de los traidores, o incluso puede existir un empate en la longitud de ambas cadenas de bloques. En ese caso, como se plantea en \citep{byzantine generals} las posibles acciones se pueden considerar como igual de buenas así que el empate en la cadena de bloques se puede romper con algún mecanismo fijado en el protocolo, por ejemplo tomando la decisión dada por la cadena cuyo bloque inicial se haya creado antes (suponiendo que se incluye un timestamp)

Así, hemos conseguido una posible solución para el problema de los generales bizantinos donde no se requiere la elección de un líder. Más aún, el posible problema de suplantación de identidad (de un general traidor intentando modificar el mensaje de un general leal) se resuelve mediante el uso del algoritmo ECDSA como se dijo en \ref{cap3:seguridad}.
\section{Limitaciones del Blockchain}
Como se vió en \ref{chap3:pow} dada una función hash y cierta cadena de caracteres el algoritmo de prueba de trabajo está basado en encontrar cierto nonce que unido a la cadena de entrada y al aplicarle la función hash devuelva un resultado con determinadas propiedades fijadas en el protocolo. Esto es una medida del esfuerzo computacional puesto en resolver ese problema y no es nada democrático, pues un único participante con una CPU avanzada podría igualar o superar al esfuerzo de varios nodos independientes. La idea detrás de usar este método en el blockchain, como se mencionó en \ref{cap3} es suponer que el \textit{valor intrínseco} de la red en su conjunto está relacionado directamente con el esfuerzo puesto en construir la cadena de bloques. Así, un ataque a una red, como la del Bitcoin por ejemplo, con un gran número de participantes sería muy difícil pues la potencia de cálculo combinada de sus nodos superaría a la de casi cualquier atacante. Y es justamente esta inversión en tiempo de CPU expresada en la cadena de bloques la que mide la importancia de la red.

Todo lo anterior supone que el bien en el que se basa la implementación del blockchain o el objetivo que persigue carece de valor por sí mismo. En el caso de las criptomonedas esto es cierto en general, pero otras situaciones, como la que plantea el problema de los generales bizantinos, puede no cumplirse.
En estos casos, dado que puede ser interesante para alguna entidad interferir en el consenso desde un principio, la solución pasa por recurrir a un blockchain de consorcio o privado. Los participantes en este tipo de redes solo tienen en cuenta los bloques generados por ciertos nodos que definidos en una lista (consorcio) o llegan incluso a ignorar los mensajes que no provengan de esos nodos previamente (privado). El carácter público

El consenso que se alcanza en el el blockchain es probabilístico: con cada bloque que se añade aumenta la probabilidad de que los nodos lleguen a un acuerdo \citep{blockchain consensus}. Como para los algoritmos \textit{Paxos} y \textit{Raft} hay que poner ciertas exigencias sobre el número de nodos maliciosos respecto al total de participantes para que sea probable llegar al consenso. Para el Bitcoin se suele aceptar que siempre que un atacante no posea más del 50\% del poder cálculo de la red será prácticamente imposible que pueda afectar al consenso. Aunque se llegue a un acuerdo sí puede ocurrir que un atacante consiga bloquear transacciones legítimas y alterar negativamente el funcionamiento de la red. Por ello en las diferentes implementaciones de criptomonedas se suelen adoptar medidas de seguridad adicionales en este sentido.
 
 %debemos suponer que los nodos malicioso tienen menos de $1/3$ del poder de cálculo de la red.

%Explicar el problema del doble gasto, que una soluci\'on es usar servidores (centralizaci\'on) y ver el %blockchain como una soluci\'on descentralizada


%Hablar de las diferentes versiones: proof of stake, proof of deep learning, ventajas y desventajas...

%\section{Algoritmo de prueba de trabajo} \label{proof of work}
%Esta sección y posiblemente la anterior sobran aquí, ver como tratar esto.
%Especificar este algoritmo, explicar su origen (evitar spam en email, ataques ddos)