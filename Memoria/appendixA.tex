\cleardoublepage

\chapter{Implementación del protocolo blockchain}\label{implementacion}
Siguiendo las ideas planteadas en el capítulo \ref{cap3} se ha realizado una implementación del protocolo Blockchain, donde se han puesto en práctica los conceptos teóricos que se han tratado en este trabajo. 


Esta implementación, realizada en el lenguaje de programación Python funciona como un gestor de documentos donde se aprovecha el carácter inmutable de la estructura de cadena de bloques. Los participantes enviarán información en formato de cadena de caracteres, que bien puedan ser artículos, trabajos o cualquier otra clase de texto que desean almacenar. Esta información va debidamente firmada usando cierta clave privada, incluyéndose también la clave pública que permite verificar la firma. Utilizando el algoritmo de prueba de trabajo visto en \ref{chap3:pow} los nodos transformarán esta información en bloques de forma que se pueda tener un registro que sirva para probar que un texto fue verificado y existía desde un momento de tiempo concreto.

Este código puede ser utilizado en un entorno real haciendo ciertas modificaciones. Por ejemplo, la estructura enlazada de la cadena de bloques puede ser utilizada a manera de índices consecutivos en tablas SQL (\textit{structured query language}) de forma que cada nuevo dato que se añada (\textit{inserts}) deba seguir la lógica de la base de datos.



La implementación se encuentra disponible en el repositorio de Github \url{https://github.com/lglaria/TFG}

%Algunas ideas: (esto es opcional) cada participante debería disponer de dos hilos de proceso. Uno de ellos para interactuar con los otros nodos y otro para ejecutar el algoritmo de consenso. La idea es que en caso que alguien haya resuelto ese puzzle parar nuestro bucle del algoritmo de consenso.
%La solución anterior tiene menos sentido en nuestro caso, en el que el bucle del algoritmo de prueba de trabajo tiene un propósito. Sin embargo, en otro entorno sí que podría ser útil esta implementación. Se podrían implementar ambas,
\section{Librerías utilizadas}
En la realización de este código se han utilizado las siguientes librerías de Python:
\begin{itemize}
\item \textit{multiprocessing}\cite{multiprocessing} Este paquete permite generar procesos que ofrecen concurrencia tanto local (en la misma máquina aprovechando los múltiples hilos (\textit{threads}) de los procesadores actuales), así como concurrencia remota, en caso de máquinas conectadas a través de una red.
También nos permite disponer de una cola FIFO (\textit{first in first out}) para que los diferentes procesos puedan compartir información, sin necesidad de hacer uso de ninguna primitiva de concurrencia. Este paquete funciona tanto en sistemas Unix como Windows y está disponible tanto para Python 2 como para Python 3.



\lstset{language=Python}
\lstset{frame=lines}
\lstset{caption={Ejemplo generación de un proceso que muestra por pantalla \textit{"hello world"}}}
\lstset{label={lst:multi_process}}
\lstset{basicstyle=\footnotesize}
\begin{lstlisting}[title=Generación de nuevos procesos]
from multiprocessing import Process

def func(argument):
    print('hello', argument)

if __name__ == '__main__':
    p = Process(target=func, args=('world',))
    p.start()
    p.join()
\end{lstlisting}

\item \textit{hashlib} \cite{hashlib} Esta librería implementa diversas funciones hash, entre las que se encuentran: SHA1, SHA224, SHA256, SHA384, y SHA512 entre otras. Para cada función hash invocada se crea un objeto con el mismo tipo de interfaz, por ejemplo si se invoca a \textit{sha256()} se creará un objeto de la función hash SHA256 para el que podrán por ejemplo calculares los valores hash que devuelve al aplicarle esta función a cierta cadena de caracteres.

\lstset{language=Python}
\lstset{frame=lines}
\lstset{caption={Ejemplo uso de hashlib}}
\lstset{label={lst:hashlib}}
\lstset{basicstyle=\footnotesize}
\begin{lstlisting}[title=Hashlib]
import hashlib
"""
Hex output of sha-224 for message m
"""
m = "Message"
hashlib.sha224(m).hexdigest()
\end{lstlisting}

\item \textit{fastecdsa} \cite{fastecdsa} Este paquete implementa protocolos criptográficos basados en curvas elípticas tal y como se vieron en el Capítulo \ref{cap1} 

Para ser utilizada esta librería debe ser instalada previamente, pues no viene por defecto en las diferentes distribuciones de Python. Puede instalarse por ejemplo mediante el comando textit{pip install fastecdsa}

\lstset{language=Python}
\lstset{frame=lines}
\lstset{caption={Generación claves pública y privada usando fastecdsa}}
\lstset{label={lst:fastecdsa_keys}}
\lstset{basicstyle=\footnotesize}
\begin{lstlisting}[title=Generación de claves]
from fastecdsa import keys, curve

"""The reason there are two ways to generate a 
keypair is that generating the public key requires
a point multiplication, which can be expensive. 
That means sometimes you may want to delay
generating the public key until it is actually needed."""

# generate a keypair (both keys) for curve P256
priv_key, pub_key = keys.gen_keypair(curve.P256)

# generate a private key for curve P256
priv_key = keys.gen_private_key(curve.P256)

# get the public key corresponding to the private key we just generated
pub_key = keys.get_public_key(priv_key, curve.P256)
\end{lstlisting}


\lstset{language=Python}
\lstset{frame=lines}
\lstset{caption={Firma y verificación de mensajes}}
\lstset{label={lst:fastecdsa_sign}}
\lstset{basicstyle=\footnotesize}
\begin{lstlisting}[title=Firma y verificación]
from fastecdsa import curve, ecdsa, keys
from hashlib import sha384

m = "a_message_to_sign"  # some message

''' use default curve and hash function (P256 and SHA2) '''
private_key = keys.gen_private_key(curve.P256)
public_key = keys.get_public_key(private_key, curve.P256)
# standard signature, returns two integers
r, s = ecdsa.sign(m, private_key)
# should return True as the signature we just generated is valid.
valid = ecdsa.verify((r, s), m, public_key)

''' specify a different hash function to use with ECDSA '''
r, s = ecdsa.sign(m, private_key, hashfunc=sha384)
valid = ecdsa.verify((r, s), m, public_key, hashfunc=sha384)

''' specify a different curve to use with ECDSA '''
private_key = keys.gen_private_key(curve.P224)
public_key = keys.get_public_key(private_key, curve.P224)
r, s = ecdsa.sign(m, private_key, curve=curve.P224)
valid = ecdsa.verify((r, s), m, public_key, curve=curve.P224)
\end{lstlisting}

\end{itemize}

\section{Características de la implementación}
El algoritmo de consenso elegido ha sido el de prueba de trabajo, combinado con la elección de la cadena de bloques más larga en caso de tener varias alternativas disponibles.

Los bloques de la cadena son arrays de Python y almacenan (en ese orden):
\begin{itemize}
\item El valor hash del bloque anterior en formato hexadecimal. Excepto para el bloque inicial al que se le asigna por defecto el valor cero en este campo.
\item La información que se desea guardar así como la firma y la clave pública de quien ha enviado (o validado) este texto. En el bloque inicial se pone el valor cero.
\item Una marca temporal, se ha usado el sistema UNIX timestamp
\item Un índice incremental. Empieza en cero en el bloque inicial.  
\item El nonce del bloque. Cadena de 32 bits a partir del algoritmo de prueba de trabajo. Este algoritmo ha sido implementado de forma similar a lo visto en \ref{chap3:pow}
\end{itemize}



\section{Aspectos técnicos.}
En esta implementación se ha utilizado la versión 3 del lenguaje de programación Python. Para el proceso de firma basado en el algoritmo ECDSA se ha usado la curva elíptica P-256 del Instituto Nacional de Estándares y Tecnología (NIST) dependiente del Departamento de Comercio de los Estados Unidos. El algoritmo de firma basado en la curva anterior, y obtenido a partir de la librería \textit{fastecdsa} cumple con los requisitos de seguridad mencionados en el capítulo \ref{cap1}.

En la implementación del algoritmo de prueba de trabajo se ha utilizado la función hash SHA-256 que ya fue mencionada en la sección \ref{chap3:pow}. Esta función, ampliamente utilizada en diversas implementaciones del Blockchain se ha obtenido a través de la librería \textit{hashlib} \citep{hashlib}. Los nodos calculan el nonce generando valores aleatorios de 32 bits. Por defecto se ha puesto una dificultad fija, pero este valor puede ser modificado y calculado a partir de una función tal y como ocurre en el Bitcoin. En el cálculo del nonce no se modifica el timestamp.
\lstset{language=Python}
\lstset{frame=lines}
\lstset{caption={Algoritmo de prueba de trabajo}}
\lstset{label={lst:code_direct}}
\lstset{basicstyle=\footnotesize}
\begin{lstlisting}[title=Algoritmo de prueba de trabajo]
def pow(self,bloque): #algoritmo de prueba de trabajo
  bloque_str = ''.join(str(e) for e in bloque)
  while True:
    var = format(random.getrandbits(32), '08b') #buscamos nonces de 32 bits
    hash_value = sha256((bloque_str+str(var)).encode('utf-8')).hexdigest()
    if hash_value[:self.longitud_nonce] == '0'*self.longitud_nonce:
      break
  bloque.append(var)
  return bloque
\end{lstlisting}


En esta implementación cada nodo almacena el conjunto de las operaciones realizadas y no la raíz Merkle, pero este punto puede ser modificado en aras de la compresión de la información.

Como ya se mencionó la red p2p sobre la que funciona el protocolo se ha construido usando la librería \textit{multiprocessing} \citep{multiprocessing}. Quedando así solucionados los problemas de concurrencia que podrían aparecer al mantener activos al mismo tiempo dos procesos que trabajan sobre la misma estructura de datos: uno para recibir los mensajes que le llegan a un usuario (nodo) y otro para ejecutar el algoritmo de consenso, actualizar la cadena y transmitir información.

Cada nodo es capaz de calcular un nuevo bloque, validar un bloque recibido, validar una cadena completa, incluir nuevos bloques previamente verificados en la cadena existente y transmitir la cadena a otros nodos. Igualmente existe un mecanismo para generar el bloque inicial, en caso que se quiera construir la estructura desde cero. Todas estas funcionalidades se encuentran implementadas como métodos de la clase Blockchain que es invocada por cada nodo.

\lstset{language=Python}
\lstset{frame=lines}
\lstset{caption={Fragmento del código de la clase Blockchain, se ha incluido el método para generar el bloque inicial}}
\lstset{label={lst:code_blockchain}}
\lstset{basicstyle=\footnotesize}
\begin{lstlisting}[title=Clase Blockchain]

class Blockchain(object): 
	"""
	formato bloques [hash_bloque_anterior, (data,data), timestamp, indice, nonce]
	"""
	def __init__(self):
		self.cadena = []
		self.longitud_nonce = 4 #dificultad algoritmo pow (modificable)

	def primer_bloque(self):
	"""
	Generacion primer bloque
	"""
		primer_bloque = [0,0,int(round(time.time())),0]
		self.cadena.append(self.pow(primer_bloque))
\end{lstlisting}

%Esta implementación puede igualmente ser aplicada en un entorno real basado por ejemplo en bases de datos de tipo SQL. En este supuesto se podria utilizar el valor hash de cada bloque como referencia de las entradas de la tabla.


\section{Seguridad de la implementación y posibles mejoras.}
Queda por hacer un análisis similar al de la sección \ref{cap3:seguridad} para esta implementación en concreto, ignorando los supuestos de privacidad que carecen de sentido en este caso. La fortaleza del algoritmo de cifrado y de la función hash dependen de las librerías elegidas para usar estos métodos y de acuerdo con la documentación no tienen problemas de seguridad conocidos.
Hay que ver por tanto si el consenso que busca alcanzar este implementación es vulnerable a algún tipo de ataque. 


Un posible agresor podría intentar realizar dos tipos de acciones que alteren el consenso:

\begin{enumerate}
\item Incluir en la cadena información incorrecta.  
\item Modificar la cadena para incluir nuevos bloques en determinada posición (distinta de la última) o modificar bloques existentes.
\end{enumerate}

Dado que este protocolo no dispone de un mecanismo para distinguir cuando un texto es correcto o no, y como en general tales mecanismos automáticos resultan difíciles de modelar una posible solución es hacer que esta implementación funcione como blockchain de consorcio o privado. Esto tiene sentido pues en una situación real quienes se encargan de verificar que un documento o trabajo cumple con ciertos criterios son exclusivamente las personas autorizadas a ello. De esta forma, aunque se pueda incluir información incorrecta al final de la cadena esto será responsabilidad de quien la haya firmado. La lista de claves públicas autorizadas a realizar validaciones se podría añadir como un campo más en la cadena de bloque o mantener una estructura de datos alternativa donde se almacene esta lista. Además de verificar que la firma se corresponde con la clave pública habría también que comprobar también que esta clave está autorizada a verificar documentos.

El segundo tipo de ataque como se vio en el capítulo \ref{cap3} depende de la dificultad de la función hash y de la longitud de la cadena. Un atacante que quiera modificar algún bloque sabemos debería rehacer toda la cadena, pues los valores hash que señalan al bloque anterior y por tanto el nonce del algoritmo de prueba de trabajo habría que volverlos a calcular. Pero ahora no dependemos solamente de la dificultad de esta operación (que por sí sola garantizaría la seguridad) sino que se puede aprovechar el hecho de disponer de un conjunto definido de claves públicas autorizadas a validar. Si además de firmar el texto hacemos que haya que firmar el valor hash del bloque anterior, un posible atacante debería disponer de alguna clave privada y además de eso la cadena falsa que genere se caracterizará por incluir a partir de la posición donde ha realizado la modificación una única firma, lo que hace muy fácil identificar esta clase de ataques.


\section{Cálculos sobre el algoritmo de prueba de trabajo}
Cuando se explicó el funcionamiento del algoritmo de prueba de trabajo se hizo referencia a la relación entre la longitud del puzzle (la condición que se le impone al valor hash que se considere válido) y al tiempo de procesamiento necesario para obtener este valor. Dado que ahora se dispone de una implementación de este algoritmo es posible comprobar este hecho en la práctica.

Se incluye una tabla con los tiempos medios que se tardado en obtener el nonce según la dificultad para ciertas cadenas generadas de forma aleatoria.
\begin{table}[htb]
\centering
\begin{tabular}{|l|l|l|}

\hline
Dificultad (ceros del hash)  & Tiempo medio (segundos) & Número de iteraciones \\
\hline \hline
1 & 9.87052917e-05 & 100\\ \hline
2 & 1.49291515e-03 & 100\\ \hline
3 & 3.24990416e-02 & 100\\ \hline
4 & 4.12933254e-01 & 100\\ \hline
5 & 6.35731197e+00 & 100\\ \hline
6 & 9.97811596e+01 & 10 \\ \hline
\end{tabular}
\caption{Resultados algoritmo prueba de trabajo.}
\label{tabla:anchofijo}
\end{table}

Aunque estos valores dependen de la máquina en la que se efectúen los cálculos, y en este caso por limitaciones técnicas no ha sido posible realizar más comprobaciones, puede apreciarse que hay relación directa entre la longitud de la cadena pedida como solución del problema y el tiempo necesario para encontrarla.

\begin{figure}[H]
%\centering
  \includegraphics[width=12cm]{figures/resultados_pow.eps}
  \caption{Comparación entre los resultados y la función $y = 2^x$}
  \label{fig:resultados}
\end{figure}
En la figura \ref{fig:resultados} se aprecia que los resultados obtenidos tienen un comportamiento similar a la función $y(x) = 2^x$. Aunque esto no puede considerarse como una aproximación buena del algoritmo de prueba de trabajo con la función hash SHA-256, pues ni el tamaño de la muestra es lo suficientemente grande, ni se han hecho comprobaciones para dificultades superiores a 6, sí es un indicador de que este comportamiento es cuasi-exponencial.
%Implementaremos el algoritmo visto en \ref{chap3:pow}
%Uno de los problemas del algoritmo de prueba de trabajo del Bitcoin es que todo el esfuerzo en términos de procesamiento que se usa para resolver el puzzle es en esencia inútil (no es verdaderamente inútil, pues cumple la función para la que está diseñado...). Esto está detrás de las graves implicaciones medioambientales del Bitcoin y de otras populares criptomonedas (poner referencia y datos sobre el consumo energético). Lo que proponemos es utilizar los cálculos con las funciones hash en el algoritmo de prueba de trabajo para buscar colisiones en la función SHA-256, tal como se definieron en \ref{hash}